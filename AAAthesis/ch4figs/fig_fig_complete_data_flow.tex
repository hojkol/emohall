\begin{figure}[h]
    \centering
    \begin{tikzpicture}[
        scale=0.8,
        node distance=1.5cm,
        auto,
        font=\small,
        flow_box/.style={rectangle, draw, thick, minimum width=2.5cm, minimum height=0.6cm,
                         text centered},
        process/.style={rounded rectangle, draw, thick, minimum width=2.5cm,
                        minimum height=0.6cm, text centered},
        arrow/.style={->, thick}
    ]

    % 第一行:用户和前端
    \node[flow_box, fill=cyan!20] (user) at (1, 7) {用户};
    \node[process, fill=cyan!30] (upload) at (1, 5.5) {上传图像\\和音频};

    % 第二行:前端处理
    \node[process, fill=green!30] (validate) at (1, 4) {参数验证\\文件检查};
    \node[process, fill=green!30] (request) at (3.5, 4) {HTTP POST\\请求};

    % 第三行:API 层
    \node[process, fill=yellow!30] (api_process) at (3.5, 2.5) {API 验证\\创建任务};
    \node[process, fill=yellow!30] (queue) at (6, 2.5) {加入\\任务队列};

    % 第四行:后台处理
    \node[process, fill=orange!30] (preprocess) at (1.5, 1) {预处理};
    \node[process, fill=orange!30] (inference) at (3.5, 1) {推理};
    \node[process, fill=orange!30] (postprocess) at (5.5, 1) {后处理};

    % 第五行:输出
    \node[process, fill=red!30] (save) at (3.5, -0.5) {保存视频};
    \node[flow_box, fill=cyan!20] (download) at (5.5, -0.5) {用户下载};

    % 连接箭头
    \draw[arrow] (user) -- (upload);
    \draw[arrow] (upload) -- (validate);
    \draw[arrow] (validate) -- (request);
    \draw[arrow] (request) -- (api_process);
    \draw[arrow] (api_process) -- (queue);
    \draw[arrow] (queue) -- (preprocess);
    \draw[arrow] (preprocess) -- (inference);
    \draw[arrow] (inference) -- (postprocess);
    \draw[arrow] (postprocess) -- (save);
    \draw[arrow] (save) -- (download);

    % 回程箭头(实时更新)
    \draw[dashed, arrow] (queue.north) -- (1.5, 3) -- (1, 3.5) [out=90];
    \node[font=\tiny] at (0.5, 3.2) {轮询};

    \node[text centered] at (3.5, 8) {\textbf{完整的数据流程}};

    \end{tikzpicture}
    \caption{用户请求到视频输出的完整数据流程}
    \label{fig:complete_data_flow}
\end{figure}
