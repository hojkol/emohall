\begin{figure}[h]
    \centering
    \begin{tikzpicture}[
        scale=1,
        node distance=2cm,
        auto,
        stage_box/.style={rectangle, draw=thick, fill=blue!20, minimum width=2.2cm,
                          minimum height=1cm, text centered},
        process_box/.style={rectangle, draw, fill=green!20, minimum width=1.8cm,
                            minimum height=0.5cm, text centered, font=\tiny},
        arrow/.style={->, thick, line width=1.5pt}
    ]

    % 标题
    \node[text centered] at (6, 5.5) {\Large \textbf{Hallo2 推理管道三阶段流程}};

    % 输入
    \node at (0.8, 4) {\textbf{输入}};
    \node[process_box] (img) at (0.8, 3.3) {人物图像};
    \node[process_box] (audio) at (0.8, 2.5) {音频文件};

    % 阶段1:预处理
    \node[stage_box, fill=cyan!30] (stage1) at (3, 3) {
        \textbf{预处理}\\
        \small Preprocessing
    };
    \node[process_box, fill=cyan!20] at (3, 1.8) {人脸检测\\掩码生成};
    \node[process_box, fill=cyan!20] at (3, 1.2) {音频分离\\特征提取};

    % 箭头
    \draw[arrow] (img) -- (1.8, 3);
    \draw[arrow] (audio) -- (1.8, 2.5);
    \draw[arrow] (1.8, 3) -- (3, 3);

    % 阶段2:推理
    \node[stage_box, fill=orange!30] (stage2) at (6, 3) {
        \textbf{推理}\\
        \small Inference
    };
    \node[process_box, fill=orange!20] at (6, 1.8) {VAE编码\\条件融合};
    \node[process_box, fill=orange!20] at (6, 1.2) {扩散逆向\\生成视频};

    \draw[arrow] (stage1) -- (stage2);

    % 阶段3:后处理
    \node[stage_box, fill=red!30] (stage3) at (9, 3) {
        \textbf{后处理}\\
        \small Postprocessing
    };
    \node[process_box, fill=red!20] at (9, 1.8) {VAE解码\\视频编码};
    \node[process_box, fill=red!20] at (9, 1.2) {音频混合\\文件输出};

    \draw[arrow] (stage2) -- (stage3);

    % 输出
    \node at (11.2, 3) {\textbf{输出}};
    \draw[arrow] (stage3) -- (11.2, 3.3);
    \node[process_box, fill=green!40] at (11.2, 2.5) {MP4视频文件};

    \end{tikzpicture}
    \caption{Hallo2 推理管道三阶段流程(预处理-推理-后处理)}
    \label{fig:hallo2_pipeline}
\end{figure}
