\begin{figure}[h]
    \centering
    \begin{tikzpicture}[
        scale=1,
        node distance=2cm,
        auto,
        state/.style={circle, draw, thick, minimum width=1.5cm, text centered},
        transition/.style={->, thick, bend left},
        font=\small
    ]

    % 标题
    \node[text centered] at (3.5, 4.5) {\large \textbf{任务生命周期状态机}};

    % 状态节点
    \node[state, fill=yellow!40] (pending) at (1, 2.5) {Pending\\(等待)};
    \node[state, fill=cyan!40] (running) at (3.5, 2.5) {Running\\(运行)};
    \node[state, fill=green!40] (completed) at (6, 2.5) {Completed\\(完成)};
    \node[state, fill=red!40] (failed) at (6, 0.5) {Failed\\(失败)};

    % 状态转移
    \draw[transition] (pending) to node[above] {submit} (running);
    \draw[transition] (running) to node[above] {success} (completed);
    \draw[transition] (running) to node[right] {error} (failed);
    \draw[transition] (pending) to node[left] {cancel} (failed);

    % 说明文字
    \node[text centered, font=\tiny] at (1, 1.5) {API已接受\\等待执行};
    \node[text centered, font=\tiny] at (3.5, 1.5) {推理进行中\\实时更新进度};
    \node[text centered, font=\tiny] at (6, 1.5) {视频已生成\\可下载};
    \node[text centered, font=\tiny] at (6, -0.5) {推理失败\\保存错误日志};

    \end{tikzpicture}
    \caption{任务生命周期状态机}
    \label{fig:task_state_machine}
\end{figure}
