\begin{figure}[h]
    \centering
    \begin{tikzpicture}[
        scale=0.9,
        node distance=1.5cm,
        auto,
        font=\small,
        box/.style={rectangle, draw, thick, minimum width=7cm, minimum height=0.8cm,
                    text centered, fill=blue!10},
        title/.style={text centered, font=\large\bfseries},
        arrow/.style={->, thick, line width=1.5pt}
    ]

    % 标题
    \node[title] at (3.5, 9) {系统五层分层架构设计};

    % 展示层
    \node[box, fill=cyan!20] (presentation) at (3.5, 7.5) {
        \textbf{展示层(Presentation Layer)}\\
        Streamlit Web UI (Port 8501)
    };

    % HTTP 通信
    \node[text centered] at (3.5, 6.8) {$\Downarrow$ HTTP REST 通信 $\Downarrow$};

    % API 层
    \node[box, fill=green!20] (api) at (3.5, 6) {
        \textbf{API 层(API Layer)}\\
        FastAPI + Uvicorn (Port 8001)
    };

    % 函数调用
    \node[text centered] at (3.5, 5.3) {$\Downarrow$ 函数调用 $\Downarrow$};

    % 服务层
    \node[box, fill=yellow!20, minimum width=8cm] (service) at (3.5, 4.5) {
        \textbf{服务层(Service Layer)}\\
        TaskManager | StateManager | ModelRegistry
    };

    % 函数调用
    \node[text centered] at (3.5, 3.8) {$\Downarrow$ 函数调用 $\Downarrow$};

    % 推理层
    \node[box, fill=orange!20, minimum width=8cm] (inference) at (3.5, 3) {
        \textbf{推理层(Inference Layer)}\\
        Hallo2 Pipeline | 数据处理器
    };

    % 函数调用
    \node[text centered] at (3.5, 2.3) {$\Downarrow$ 函数调用 $\Downarrow$};

    % 模型层
    \node[box, fill=red!20, minimum width=8cm] (model) at (3.5, 1.5) {
        \textbf{模型层(Model Layer)}\\
        PyTorch + CUDA + 预训练模型
    };

    % 右侧说明
    \node[text centered, anchor=west] at (11.5, 7.5) {\small \textbf{用户交互}};
    \node[text centered, anchor=west, font=\tiny] at (11.5, 7.1) {文件上传、参数配置};

    \node[text centered, anchor=west] at (11.5, 6) {\small \textbf{请求处理}};
    \node[text centered, anchor=west, font=\tiny] at (11.5, 5.6) {参数验证、路由};

    \node[text centered, anchor=west] at (11.5, 4.5) {\small \textbf{业务逻辑}};
    \node[text centered, anchor=west, font=\tiny] at (11.5, 4.1) {任务管理、状态管理};

    \node[text centered, anchor=west] at (11.5, 3) {\small \textbf{核心推理}};
    \node[text centered, anchor=west, font=\tiny] at (11.5, 2.6) {模型执行、数据处理};

    \node[text centered, anchor=west] at (11.5, 1.5) {\small \textbf{硬件基础}};
    \node[text centered, anchor=west, font=\tiny] at (11.5, 1.1) {深度学习框架};

    \end{tikzpicture}
    \caption{数字人视频生成系统五层分层架构}
    \label{fig:system_architecture_layers}
\end{figure}
